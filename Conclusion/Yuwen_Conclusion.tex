 \chapter{Conclusion} \label{Yuwen_Conclusion}
This thesis aims to develop a consistent method which integrates volumetric CT data, quantitative analysis, pulmonary functional test and computational modeling to investigate the lung function and disease progression of IPF. This chapter provides a summary of the important outcomes of the research, how the thesis addressed the objectives set out in Chapter 1, Section \ref{ThesisObjective}, and makes some plans for the future work.

% Section 1
\section{Thesis Summary} \label{ThesisSummary}
\textit{\paragraph{Objective 1:} Develop an automatic lung lobe segmentation method based on HRCT images that can consistently estimate lobar boundaries even if the boundary is not clear along its entire length. The method needs to be tested on both healthy subjects and IPF subjects. Ideally the algorithm will be automatic , or semi-automatic.}

\noindent  A PCA-based statistical shape model guided method was developed to segment pulmonary lobes from HRCT images in Chapter 3. The method was tested on 20 inspiratory and expiratory CT scans: 10 healthy young subjects and 10 older subjects diagnosed with IPF. Results show that the method can perform well to detect the location of the fissures over most of the fissure surfaces on CT images from normal subjects, and provides a relatively accurate result (compared with some existing tools) for IPF (abnormal) subjects although manual interaction is still needed for a few subjects. The method provided more accurate result on the left oblique fissure than the other two fissures, because the left lung has a simpler anatomic structure with only one fissure. The ability of the method to provide an initial estimate of the fissure locations was compared with two other segmentation softwares which use watershed-based (anatomical knowledge-based) methods. The two segmentation softwares PASS and PTK were unable to segment 9/20 and 7/20 subjects respectively. In contrast, our model-based method can estimate the fissure location for all subjects at all volumes, although PASS can perform better on some young healthy subjects. This new procedure does not depend on prior segmentation of anatomical structures (airway lobar classification) and has promising potential as a clinically useful semi-automatic lobe segmentation procedure. 
\textit{\paragraph{Objective 2:} Classify and quantify tissue abnormalities from HRCT scans of IPF lungs. Analyze the density, volume, spatial distribution, and their change over time.}

\noindent The tissue abnormalities of IPF lungs were classified and analyzed over time using quantitative methods based on HRCT imaging in Chapter 4. Tissue regions of HRCT images were classified using CALIPER software. The classified data was mapped to a statistical shape model, and quantitative approaches was used to analyze tissue density, tissue volume, the spatial distribution of abnormalities, and regional changes in tissue over time. The result shows that tissue density of different tissue CT patterns fluctuate within different ranges over time. The ground-glass region has the highest average tissue density, whereas emphysema has the lowest average tissue density. Therefore, tissue density can be used as a quantitative biomarker to distinguish the disease regions from other lung parenchyma. However, tissue volume can not be used as a reliable biomarker to interpret the progression of disease, as no regular tendency of the volume change for abnormal tissues (honeycomb, reticular and ground-glass) was observed. It is also illustrated that ground-glass mostly locates in the basal part of lung (lower lobe), while emphysema usually presents in the apex part (upper lobe). The distribution of reticular region mainly focuses on the basal area and apex area. All the abnormalities (emphysema and fibrosis) in IPF lung are peripheral performance and mostly distributes surrounding the surface of the lung. Additionally, the locations of abnormalities in IPF lung keep changing all the time. One kind of disease pattern may change to other patterns over time. The quantitative analysis would potentially provide consistent and reliable tissue-level markers to help with further modeling of ventilation/perfusion mismatch and impaired gas exchange.
\textit{\paragraph{Objective 3:} Investigate the difference of lung shape between IPF and older normal lungs, and explore the correlation of lung shape change in IPF with the extent fibrosis.}

\noindent In Chapter 4, the geometric mesh of IPF lung and lobes were compared with a statistical shape model of old normal to quantify the lung lobe shape difference between IPF patient and old normal people, and the impact of fibrosis extent on lung shape variation was also investigated. It is found that the first shape mode of SSM (based on PCA) is significantly different between IPF subjects and old normal subjects and strongly correlates with the percentage of fibrosis. The first shape mode corresponds mainly to the anteroposterior diameter of lung which results in a variation of diaphragm. The basal and peripheral performance of disease may increase the stiffness of the lower part of lung, thus having a impact on the movement of diaphragm when breathing. Fibrosis lesion increases the ratio of anteroposterior diameter to the height of lung, which makes lung become ''fatter'' and ''shorter''. Moreover, lobe volume analysis shows that IPF lung has a lower average volume proportion for left lower lobe and left upper lobe compared with old normal lungs. This ''compression effect'' on the basal part of IPF lung may be associated the reduction in tissue compliance of IPF lungs caused by fibrosis.
\textit{\paragraph{Objective 4:} Integrate the image-based tissue quantification, pulmonary functional test and computational modeling to simulate the lung function in IPF, compared with the older normal people, and to use these tools to estimate the impact of CT features of IPF on lung function.}

\noindent In Chapter 5, a multiple time point patient-specific computational modeling method was developed with CT feature data and PFT result as input to investigate the lung geometry and lung function in IPF patient. The lung function of V/Q distribution and gas exchange were compared between IPF patient and older normal people. Compared with the older normal people, IPF patient has a lower $\mathrm{PaO_2}$ and a higher V/Q ratio, and it can be seen a decline of lung function in IPF lung over time, with decreasing $\mathrm{PaO_2}$ and more severe V/Q mismatch, due to the increasing stiffness of the lung. From the modeling result, the abnormalities on volumetric CT imaging are probably not sufficient to explain the decline of lung function in IPF patient, and the impaired gas exchange (V/Q mismatch) will appear in not only abnormal region but also in CT-classified 'normal' tissues. The result shows the developed patient-specific modeling method can be used to simulate the ventilation, perfusion and gas exchange in IPF lung, and is able to potentially provide auxiliary information for clinicians to assist with the radiological and biopsy diagnosis at the early stage of IPF.

% Section 1
\section{Future directions}\label{FutureDirection}
\subsection{Improvement of statistical shape model}
The SSM used in this thesis was generated using 35 old healthy subjects (AGING) and 15 young healthy subjects (HLA). The main limitation of this model is the small number of HLA subjects used in the training set, and there is a age gap between AGING group and HLA group. In order to capture more kinds of components of shape variation, a larger number of subjects with a wider range of age should be added into the SSM in the future work. With adding more subjects, it is possible to generate a more sophisticated model describing the continuous shape variation from young to old, which may increase the accuracy for estimating the initial fissure locations. Furthermore, the construction of SSM of IPF patients could also be taken into consideration for the future work. In Chapter 4, it has been proved that there is a significant shape difference between IPF and older normal lungs. Therefore, a SSM specific to IPF patients may provide a way to identify shape features of IPF patients, and would be able to provide a better fissure guessing for IPF subjects and further help with statistical analysis of IPF lung function. 

\subsection{Involvement of diffusing capacity for carbon monoxide in the lung functional model}
It has been found that DLCO is reduced in 98\% of IPF patients at the initial diagnosis \citep{cortes2014idiopathic}, and some study indicated that the impairment of diffusion capacity is one of the reasons to cause chronic arterial hypoxaemia in IPF lung \citep{plantier2018physiology}. DLCO is an important measurement in PFT report that can help clinicians to make a diagnosis of IPF, thus could be used as a bio-marker in our computational modeling of IPF lung function in the future work. The value of DLCO is determined by the structural and functional properties of the lung \citep{graham20172017}. The structural properties includes lung gas volume, thickness and area of the alveolar capillary membrane, and the volume of hemoglobin in capillaries. The functional properties are mainly related to absolute levels of ventilation and perfusion, the uniformity of the distribution of ventilation relative to the distribution of perfusion, and the concentration and binding properties of hemoglobin in the alveolar capillaries \citep{graham20172017}. In IPF lung, there will be a reduction in the surface area for diffusion because of emphysema, and a increase in the thickness of the alveolar capillary membrane due to fibrosis. Therefore, through developing a DLCO calculation component in the model, we are able to parameterize the patient-specific properties of alveolar capillary membrane for fibrosis and emphysema areas using the measured DLCO from PFT result, furthermore to investigate if the abnormalities on volumetric are sufficient enough to explain increased lung stiffness and decreases in DLCO over time.   

\subsection{Relationship between lung vasculature and tissue abnormality or lung function in patients with IPF}
In Chapter 2, Section \ref{VasculatureAlterations}, it has been mentioned that vascular alterations are observed in the pulmonary vasculature in patients with IPF, however the vascular factor has not been involved in the modeling of this thesis. The study of \cite{Jacob2016Evaluation,Jacob2016Mortality} has pointed out that there is a increase in the pulmonary vessel volume (PVV) in IPF lung, and PVV has strong links with ILD extent (includes ground glass, reticular and honeycomb patterns) ($R^2 = 0.73$, $P < 0.0001$) by using linear regression analysis. Moreover, PVV was demonstrated to be an independent predictor of mortality and a stronger predictor of mortality than the other traditional CT features and pulmonary functional variables, such as FVC, DLCO, carbon monoxide transfer coefficient and composite physiologic index. Therefore, in the next stage, it would be worthwhile to explore the relationship between the alterations in lung vasculature and tissue abnormality or lung function of IPF. The distribution of PVV in each lung and each lobe could be calculated, and then whether PVV by lung or lobe is associated with regional tissue abnormality or lung function would be tested. Meanwhile, a vascular tree model could be developed to simulate perfusion and vascular distension with subject-specific PVV and tissue abnormalities. Another future direction is to use multiple time point data to further study the association of the change in PVV over time with the progression of IPF.
