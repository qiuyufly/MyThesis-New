 \chapter{Conclusion} \label{Yuwen_Conclusion}
This thesis aimed to develop a consistent method with which to integrate volumetric CT data, quantitative analysis, pulmonary function test data and computational modelling to investigate structure-function relationships in IPF. This chapter provides a summary of the important outcomes of the research, how the thesis addressed the objectives set out in Chapter 1, Section \ref{ThesisObjective}, and makes some suggestions for future work.

% Section 1
\section{Thesis Summary} \label{ThesisSummary}
\textit{\paragraph{Objective 1:} Develop an automatic lung lobe segmentation method based on HRCT images that can consistently estimate lobar boundaries even if the boundary is not clear along its entire length. The method needs to be tested on both healthy subjects and IPF subjects. Ideally the algorithm will be automatic , or semi-automatic.}

\noindent  A PCA-based statistical shape model guided method was developed to segment pulmonary lobes from HRCT images in Chapter 3. The method was tested on 20 inspiratory and expiratory CT scans: 10 healthy young subjects and 10 older subjects diagnosed with IPF. Results show that the method can perform well to detect the location of the fissures over most of the fissure surfaces on CT images from normal subjects, and provides a relatively accurate result (compared with some existing tools) for IPF (abnormal) subjects although manual interaction is still needed for a few subjects. The method provided a more accurate result on the left oblique fissure than the other two fissures, because the left lung has a simpler anatomic structure with only one fissure. The ability of the method to provide an initial estimate of the fissure locations was compared with two other segmentation softwares which use watershed-based (anatomical knowledge-based) methods. The two segmentation softwares PASS and PTK were unable to segment 9/20 and 7/20 subjects respectively. In contrast, our model-based method can estimate the fissure location for all subjects at all volumes. Although PASS can provide a more accurate segmentation for some young healthy subjects, the new method was on average as accurate as PASS, and more accurate than PTK. This new procedure does not depend on prior segmentation of anatomical structures (airway lobar classification) and has promising potential as a clinically useful semi-automatic lobe segmentation procedure. 
\textit{\paragraph{Objective 2:} Classify and quantify tissue abnormalities from HRCT scans of IPF lungs. Analyse the density, volume, spatial distribution, and their change over time.}

\noindent The tissue abnormalities of IPF lungs were classified and analysed at several time points using quantitative methods based on HRCT imaging in Chapter 4. Tissue regions of HRCT images were classified using CALIPER software. The classified data was mapped to a statistical shape model, and quantitative approaches were used to analyse tissue density, tissue volume, the spatial distribution of abnormalities, and regional changes in tissue over time. Spatial distribution of tissue abnormalities was quantitatively described in apical-basal, anterior-posterior, internally-to-externally, and by lobe. The distributions of fibrosis were generally consistent between subjects and did not change over time (although the amount of fibrosis increased somewhat).
\textit{\paragraph{Objective 3:} Quantify the difference of lung shape between IPF and older normal lungs, and explore the correlation of lung shape change in IPF with the extent of fibrosis.}

\noindent In Chapter 4, the geometric mesh of IPF lung and lobes were compared with a statistical shape model of old normal to quantify the lung lobe shape difference between IPF patients and healthy subjects aged $>$ 50 years, and the impact of fibrosis extent on lung shape variation was also investigated. It was found that the first shape mode of the SSM (based on PCA) is significantly different between IPF subjects and normal subjects and strongly correlates with the percentage of fibrosis. The first shape mode corresponds mainly to the anteroposterior diameter of the lung. While there is a difference in the shape of the IPF lung compared with normal, the shape does not appear to change with progression of the disease.
\textit{\paragraph{Objective 4:} Integrate the image-based tissue quantification, pulmonary function tests and computational modelling to simulate lung function in IPF, and compare with simulated lung function in older normal people, and use these tools to estimate the impact of radiological features of IPF on lung function.}

\noindent In Chapter 5, a multiple time point patient-specific computational modelling method was developed using CT and PFT data as input to investigate the lung geometry and lung function in two IPF patients. The simulated V/Q distribution and gas exchange, as a result of abnormal tissue function, were compared between IPF patients and normal ''control'' models. Compared with the control models, the IPF models had a lower $\mathrm{PaO_2}$ and increased V/Q mismatch. V/Q mismatch and decrease in $\mathrm{PaO_2}$ increased for successive imaging time points. The modelling results suggest that the abnormalities on volumetric CT imaging are not sufficient to explain the decline of lung function (decrease in tissue compliance) in IPF patients, and the impaired gas exchange (V/Q mismatch) will appear in not only abnormal regions but also in CT-classified 'normal' tissues. The results also suggest that V/Q mismatch as a direct result of abnormally stiff and underperfused tissue results in progressive worsening of gas exchange that mirrors the decline in DLCO. That is, thickening of the gas exchange barrier or increase in anatomical shunts are not necessary to explain gas exchange abnormality in IPF.

% Section 1
\section{Future directions}\label{FutureDirection}
\subsection{Improvement of the statistical shape model}
The SSM used in this thesis was generated using 35 healthy subjects aged $>$ 50 years. In order to capture more kinds of components of shape variation, a larger number of subjects with a wider range of age should be added into the SSM in the future work. With adding more subjects, it could be possible to generate a more sophisticated model describing a continuous shape variation with age, which may increase the accuracy for estimating the initial fissure locations. Furthermore, the construction of an SSM of the IPF lung could also be taken into consideration for future work. In Chapter 4, it was shown that there is a significant shape difference between IPF and older normal lungs. Therefore, a SSM specific to IPF patients may provide a better way to identify shape features of IPF patients, and would be able to provide a better fissure estimation for IPF subjects and further help with statistical analysis of IPF lung function. 

\subsection{Involvement of diffusing capacity for carbon monoxide in the lung functional model}
It has been found that DLCO is reduced in 98\% of IPF patients at the initial diagnosis \citep{cortes2014idiopathic}, and some studies indicate that the impairment of diffusion capacity is one cause of chronic arterial hypoxaemia in IPF \citep{plantier2018physiology}. DLCO is an important measurement in the PFT report that can help clinicians to make a diagnosis of IPF, thus could be used as a bio-marker in our computational modelling of the IPF lung function in future work. The value of DLCO is determined by the structural and functional properties of the lung \citep{graham20172017}. The structural properties includes lung gas volume, thickness and area of the alveolar capillary membrane, and the volume of haemoglobin in capillaries. The functional properties are mainly related to absolute levels of ventilation and perfusion, the uniformity of the distribution of ventilation relative to the distribution of perfusion, and the concentration and binding properties of haemoglobin in the alveolar capillaries \citep{graham20172017}. In the IPF lung, there will be a reduction in the surface area for diffusion, and an increase in the thickness of the alveolar capillary membrane due to fibrosis. Therefore, through developing a DLCO calculation component in the model, the patient-specific properties of the alveolar capillary membrane could be parameterized for fibrosis and emphysema areas by fitting to the measured DLCO.

%\subsection{Relationship between lung vasculature and tissue abnormality or lung function in patients with IPF}
%In Chapter 2, Section \ref{VasculatureAlterations}, it has been mentioned that vascular alterations are observed in the pulmonary vasculature in patients with IPF, however the vascular factor has not been involved in the modelling of this thesis. The study of \cite{Jacob2016Evaluation,Jacob2016Mortality} has pointed out that there is a increase in the pulmonary vessel volume (PVV) in IPF lung, and PVV has strong links with ILD extent (includes ground glass, reticular and honeycomb patterns) ($R^2 = 0.73$, $P < 0.0001$) by using linear regression analysis. Moreover, PVV was demonstrated to be an independent predictor of mortality and a stronger predictor of mortality than the other traditional CT features and pulmonary functional variables, such as FVC, DLCO, carbon monoxide transfer coefficient and composite physiologic index. Therefore, in the next stage, it would be worthwhile to explore the relationship between the alterations in lung vasculature and tissue abnormality or lung function of IPF. The distribution of PVV in each lung and each lobe could be calculated, and then whether PVV by lung or lobe is associated with regional tissue abnormality or lung function would be tested. Meanwhile, a vascular tree model could be developed to simulate perfusion and vascular distension with subject-specific PVV and tissue abnormalities. Another future direction is to use multiple time point data to further study the association of the change in PVV over time with the progression of IPF.
