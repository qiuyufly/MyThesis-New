\chapter{Introduction} \label{Yuwen_Introduction}
\section{Motivation}
Idiopathic pulmonary fibrosis (IPF) is a lethal fibrosing lung disorder that typically affects adults in the sixth to seventh decade of life \citep{meltzer2008idiopathic,king2011idiopathic}. IPF is more frequent in males than females, and is associated with environmental factors such as smoking or dust exposure, but as its name suggests does not have a clear aetiology. There is no known cure, although some very new therapies have been suggested to slow the rate of physiologic decline \citep{raghu2011official}. IPF belongs to the family of \gls{ild}, and is the most frequently diffuse occurring interstitial lung disease \citep{meltzer2008idiopathic}. In IPF lungs, healthy tissues are gradually replaced by an abnormal and excessive deposition of collagen (fibrosis), which may lead to a reduction in lung volumes, decreased lung compliance, mismatching at ventilation and perfusion, impaired gas exchange, and ultimately respiratory failure and death \citep{richeldi2017idiopathic}.

Diagnosis of IPF relates to a histopathological or radiological pattern typical of \gls{uip} \citep{raghu2011official,xaubet2017idiopathic}. UIP pattern is usually associated with honeycombing (subpleural cystic airspaces with well-defined walls),  reticular opacities and ground-glass abnormalities \citep{raghu2011official,richeldi2017idiopathic}. These abnormalities in IPF typically develop preferentially in posterior-basal lung regions, and often co-exists with emphysema, which causes a progressive and irreversible decline in lung function. Currently it is not clear how - or whether - the spatial distribution of tissue abnormalities in IPF (including classifications of tissue type) correlate with \gls{pfts}. In addition, the progression of IPF is variable between individuals, and no established quantitative tools exist to assess its development and how the tissue abnormalities change over time in this condition. Therefore, developing a computational modelling of lung function in IPF over time, together with clinical knowledge, will be helpful for clinician to have a better understanding of patient prognosis, and presents potential novel ways of investigating strategies for patient-specific diagnosis and treatment planning for IPF patients.

\section{Thesis objectives} \label{ThesisObjective}
The primary aim of this thesis is to contribute toward developing a new quantitative tool that integrates data from volumetric imaging, PFTs, and computational models for lung function, to take steps toward predicting the development of IPF over time, and to understand differences, including lung shape and lung function, between IPF and normal older lungs. 

The specific objectives of this thesis are:

\paragraph{Objective 1:} Develop an automatic lung lobe segmentation method based on HRCT images that can consistently estimate lobar boundaries even if the boundary is not clear along its entire length. The method needs to be tested on both healthy subjects and IPF subjects. Ideally the algorithm will be automatic, or semi-automatic.
\paragraph{Objective 2:} Classify and quantify tissue abnormalities from HRCT scans of IPF lungs. Analyse the density, volume, spatial distribution, and their change over time.
\paragraph{Objective 3:} Investigate the difference of lung shape between IPF and older normal lungs, and explore the correlation of lung shape change in IPF with the extent fibrosis.
\paragraph{Objective 4:} Integrate the image-based tissue quantification, pulmonary functional test and computational modeling to simulate the lung function in IPF, compared with the older normal people, and to use these tools to estimate the impact of CT features of IPF on lung function.

\section{Thesis overview}
\paragraph{Chapter 2} In order to quantitatively explore the link between structure and lung function of patients with IPF, some background knowledge of this disease is summarized in this chapter. The first part provides an basic introduction of IPF, including its epidemiology, adetiology, pathogenesis, diagnosis, clinical courses and comorbidities. The second part describes some physiological alterations in IPF lungs. The changes in the mechanical properties of the lungs in IPF and the changes in pulmonary gas exchange are discussed in details.
\paragraph{Chapter 3} Automatic identification of pulmonary lobes from imaging is important for the further image-based analysis of lung function and disease progression. In order to overcome difficulties in identifying pulmonary fissures, especially in disease, a statistical finite element shape model of lobes is applied to guide lobar segmentation in this chapter. By deforming a principal component analysis based statistical shape model onto an individual’s lung shape, the likely region of fissure locations is predicted to initialize the search region for fissures. Then, an eigenvalue of Hessian matrix analysis and a connected component eigenvector based analysis are used to determine a set of fissure-like candidate points. A smooth multi-level B-spline curve is fitted to the most fissure-like points (those with high fissure probability) and the fitted fissure plane is extrapolated to the lung boundaries. The method is tested on 20 inspiratory and expiratory CT scans, and compared with existing algorithms in healthy young subjects and older subjects with IPF.

\begin{itemize}
  \item Yuwen, Z., Mahyar O., Alys, R., Hari, K., Margaret, L.W., David, G.M., Eric, A.H., and Merryn, H.T. (2019, February). Pulmonary lobar segmentation from computed tomography scans based on a statistical finite element analysis of lobe shape. In: SPIE Medical Imaging, International conference. San Diego, USA.
\end{itemize}

\paragraph{Chapter 4} This chapter outlines the study of quantitative analysis of IPF disease features in HRCT scans, including both tissue abnormality quantification and lung lobe shape analysis, to provide consistent potential tissue-level and shape-level biomarkers that can indicate the likely progression of the disease over time. Lung tissues are classified as normal, reticular, ground glass, or emphysema using CALIPER (Computer-Aided Lung Informatics for Pathology Evaluation and Ratings) software. The classified data is then mapped to a \gls{ssm}, which allows a reliable comparison between different patients or within one patient of different time points. Quantitative approaches are used to analyse tissue density, tissue volume, the spatial distribution of abnormalities, and regional changes in tissue over time. Principal component analysis (PCA) based SSM is applied to understand lung shape differences between IPF and normal older lungs through quantifying principal modes of shape variation of both IPF and old normal subjects. 

\begin{itemize}
  \item Yuwen, Z., Alys, R., Hari, K., Brian, B.J, and Merryn, H.T. (2017, November). Quantitative analysis of idiopathic pulmonary fibrosis abnormality from CT imaging. In: 13th Engineering Mathematics and Applications Conference, International conference. Auckland, New Zealand.
	\item Yuwen, Z., Alys, R., Hari, K., David, G.M., Margaret, L.W., Brian, B.J, and Merryn, H.T. (2018, March). High resolution CT-based characterization analysis of idiopathic pulmonary fibrosis. In: the Thoracic Society of Australia \& New Zealand (TSANZ) conference, International conference. Adelaide, Australia.
\end{itemize}

\paragraph{Chapter 5} In this chapter, data from volumetric imaging, quantitative tissue-level and shape-level features, and PFT are integrated to guide a patient-specific computational models of lung function of IPF. In order to make a comparison of lung function between IPF patients and older normal people, for each patient, a subject-specific lung mesh that represents the lung shape of old normal individual with the same age, BMI and pulmonary functional data is predicted using SSM. Anatomically based airway and blood vessel trees are generated derived from HRCT images, and are matched to the IPF lung mesh and the corresponding old normal lung mesh. The ventilation, perfusion and gas exchange models are then used to simulate $\dot{V}$, $\dot{Q}$ distribution and gas transport in normal and IPF lungs.

\begin{itemize}
  \item Yuwen, Z., Mahyar O., Alys, R., Hari, K., David, G.M., Margaret, L.W., Eric, A.H., and Merryn, H.T. (2018, March). Idiopathic pulmonary fibrosis: a study using volumetric imaging and functional data in a computational lung model, International conference. San Diego, USA.
\end{itemize}

\paragraph{Chapter 6} Key findings of this thesis are summarized in this chapter, with discussion of the main methods, models and outcomes. Potential future directions that need to be addressed to develop the modelling framework are also discussed. 
