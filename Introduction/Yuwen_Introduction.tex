\chapter{Introduction} \label{Yuwen_Introduction}
\section{Motivation}
Idiopathic pulmonary fibrosis (IPF) is a lethal fibrosing lung disorder that typically affects adults in the sixth to seventh decade of life \citep{meltzer2008idiopathic,king2011idiopathic}. IPF is more frequent in males than females, and is associated with environmental factors such as smoking or dust exposure, but as its name suggests does not have a clear aetiology. There is no known cure, although some very new therapies have been suggested to slow the rate of physiologic decline \citep{raghu2011official}. IPF belongs to the family of \gls{ild}, and is the most frequently diffuse occurring interstitial lung disease \citep{meltzer2008idiopathic}. In IPF lungs, healthy tissues are gradually replaced by an abnormal and excessive deposition of collagen (fibrosis), which may lead to a reduction in lung volumes, decreased lung compliance, mismatching of ventilation and perfusion, impaired gas exchange, and ultimately respiratory failure and death \citep{richeldi2017idiopathic}.

Diagnosis of IPF relates to a histopathological or radiological pattern typical of \gls{uip} \citep{raghu2011official,xaubet2017idiopathic}. UIP pattern is usually associated with honeycombing (subpleural cystic airspaces with well-defined walls),  reticular opacities and ground-glass abnormalities \citep{raghu2011official,richeldi2017idiopathic}. These abnormalities in IPF typically develop preferentially in the posterior-basal lung regions, and often co-exist with emphysema, which causes a progressive and irreversible decline in lung function. Currently it is not clear how - or whether - the spatial distribution of tissue abnormalities in IPF (including classifications of tissue type) correlate with \gls{pfts}. In addition, the progression of IPF is variable between individuals, and no established quantitative tools exist to assess its development and how the tissue abnormality changes over time contribute to lung function. Therefore, developing a computational model of lung function in IPF that can be parameterised to different time points, presents a potential novel way of investigating strategies for patient-specific diagnosis and treatment planning for IPF patients.

\section{Thesis objectives} \label{ThesisObjective}
The broad aim of this thesis is to contribute toward developing a new quantitative tool that integrates data from volumetric imaging, PFTs, and computational models for lung function, as a step towards predicting the development of IPF over time, and to understand differences, including lung shape and lung function, between IPF and normal older lungs. 

The specific objectives of this thesis are:

\paragraph{Objective 1:} Develop an automatic lung lobe segmentation method for HRCT images that can consistently estimate lobar boundaries even if the boundary is not clear along its entire length. The method needs to be tested on both healthy subjects and IPF subjects. Ideally the algorithm will be automatic, or semi-automatic.
\paragraph{Objective 2:} Classify and quantify tissue abnormalities from HRCT scans of IPF lungs. Analyse the density, volume, spatial distribution, and their change over time.
\paragraph{Objective 3:} Quantify the difference of lung shape between IPF and older normal lungs, and explore the correlation of lung shape change in IPF with the extent of fibrosis.
\paragraph{Objective 4:} Integrate the image-based tissue quantification, pulmonary function tests and computational modelling to simulate lung function in IPF, and compare with simulated lung function in older normal people, and use these tools to estimate the impact of radiological features of IPF on lung function.

\section{Thesis overview}
\paragraph{Chapter 2} In order to quantitatively explore the link between structure and lung function of patients with IPF, some background knowledge of this disease is summarized in this chapter. The first part provides a basic introduction to IPF, including its epidemiology, aetiology, pathogenesis, diagnosis, clinical course and comorbidities. The second part describes some physiological alterations in IPF lungs. The changes in the mechanical properties of the lungs in IPF and the changes in pulmonary gas exchange are discussed in detail.
\paragraph{Chapter 3} Automatic identification of pulmonary lobes from imaging is important for image-based analysis of lung function and disease progression. In order to overcome current difficulties in identifying pulmonary fissures, especially in disease, a statistical finite element shape model of the lobes is applied to guide lobar segmentation in this chapter. By deforming a principal component analysis-based \gls{ssm} onto an individual’s lung shape, the likely region of fissure locations is predicted to initialize the search region for fissures. Then, an eigenvalue of Hessian matrix analysis and a connected component eigenvector-based analysis are used to determine a set of fissure-like candidate points. A smooth multi-level B-spline curve is fitted to the most fissure-like points (those with high fissure probability) and the fitted fissure plane is extrapolated to the lung boundaries. The method is tested on 20 inspiratory and expiratory CT scans, and compared with existing algorithms in healthy young subjects and older subjects with IPF. This chapter has been published as a conference paper:

\begin{itemize}
  \item Yuwen Zhang., Mahyar Osanlouy., Alys Clark., Haribalan Kumar., Margaret Wilsher., David Milne., Eric Hoffman., and  Merryn Tawhai. (2019, February). Pulmonary lobar segmentation from computed tomography scans based on a statistical finite element analysis of lobe shape. In: SPIE Medical Imaging, International conference. San Diego, USA.
\end{itemize}

\paragraph{Chapter 4} This chapter details a quantitative analysis of IPF disease features in HRCT scans, including both tissue abnormality quantification and lung lobe shape analysis, to provide consistent tissue-level (distribution of abnormalities) and organ-level (shape) bio-markers that can be used as additional information to track the progression of the disease over time. Lung tissues are classified as normal, reticular, ground glass, or emphysema using CALIPER (Computer-Aided Lung Informatics for Pathology Evaluation and Ratings) software. The classified data is then mapped to a \gls{ssm}, which allows a reliable comparison between different patients or within one patient at different time points. Quantitative approaches are used to analyse tissue density, tissue volume, the spatial distribution of abnormalities, and regional changes in tissue over time. A \gls{pca} based SSM is used to understand lung shape differences between IPF and the lungs of normal subjects aged $>$ 50 years through quantifying principal modes of shape variation of both IPF and normal subjects. The results of the chapter have been presented at two conferences:

\begin{itemize}
  \item Yuwen Zhang., Alys Clark., Haribalan Kumar., Brian Bartholmai, and Merryn Tawhai. (2017, November). Quantitative analysis of idiopathic pulmonary fibrosis abnormality from CT imaging. In: 13th Engineering Mathematics and Applications Conference, International conference. Auckland, New Zealand.
	\item Yuwen Zhang., Alys Clark., Haribalan Kumar., David Milne., Margaret Wilsher., Brian Bartholmai, and Merryn Tawhai. (2018, March). High resolution CT-based characterization analysis of idiopathic pulmonary fibrosis. In: the Thoracic Society of Australia \& New Zealand (TSANZ) conference, International conference. Adelaide, Australia.
\end{itemize}

\paragraph{Chapter 5} In this chapter, data from volumetric imaging, quantitative tissue-level and shape-level features, and PFTs are integrated to guide a patient-specific computational model of lung function in IPF. In order to compare lung function between IPF patients and normal older controls, for each patient, a subject-specific lung mesh that represents the lung shape of a normal individual with the same age, BMI and pulmonary function data is predicted using an SSM. Anatomically-based models of the airway and blood vessel trees are generated from the HRCT images, and are matched to both the IPF lung mesh and the corresponding normal control lung mesh. $\dot{V}$, $\dot{Q}$ and gas exchange models are then used to simulate $\dot{V}$ and $\dot{Q}$ distributions and gas transport in normal and IPF lungs. Part of this chapter has been presented as a conference poster:

\begin{itemize}
  \item Yuwen Zhang., Alys Clark., Haribalan Kumar., Margaret Wilsher., David Milne., Brian Bartholmai, and Merryn Tawhai. (2018, March). Idiopathic pulmonary fibrosis: a study using volumetric imaging and functional data in a computational lung model, International conference. San Diego, USA.
\end{itemize}

\paragraph{Chapter 6} Key findings of this thesis are summarized in this chapter, with discussion of the main methods, models and outcomes. Potential future directions that need to be addressed to develop the modelling framework are also discussed. 
